\documentclass{article}

\begin{document}

\title{Dokaz s protislovjem}
\author{Luka Uršič, 89221145}
\date{\today}

\maketitle

\section{Trditev}

Algoritem, ki razporedi (sortira) naloge glede na čas njihovega reševanja (od najkrajšega do najdaljšega) in jih nato reši v tem vrstem redu, maksimizira skupno število točk.

\section{Dokaz}

Predpostavimo, da algoritem ni optimalen, torej obstaja boljša razporeditev nalog, ki prinese več točk.

\subsection{Optimalna razporeditev}

Naj bo optimalna razporeditev nalog $O$, ki prinese več točk, kot naša razporeditev $R$. V razporeditvi $O$ naj bo naloga $i$ prva naloga, ki ni ista kot v razporeditvi $R$. V $O$ je naloga $i$ pred nalogo $j$, čeprav $i$ traja več kot $j$.

\subsection{Zamenjava}

Če bi zamenjali $i$ in $j$ nalogi v razporeditvi $O$, bi zmanjšali celoten čas od trenutka, ko rešujemo $i$, s čemer bi povečali število točk, pridobljenih od naloge $i$.

\subsection{Protislovje}

Po zamenjavi bi lahko dobili več ali enako število točk kot v $R$, kar nas privede do kontradikcije s predpostavko, da je $O$ boljša razporeditev od $R$.

\[
    \Rightarrow\!\Leftarrow
\]

\section{Zaključek}

Predpostavka torej ne drži. Algoritev v trditvi je optimalen.

\end{document}